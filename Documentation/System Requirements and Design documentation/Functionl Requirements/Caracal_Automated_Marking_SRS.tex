\documentclass{article}
%----------------------------------------------------------------------------------------
%	PACKAGES AND OTHER DOCUMENT CONFIGURATIONS
\usepackage[utf8]{inputenc}
\usepackage{graphicx}
\usepackage{tabularx}
\usepackage[export]{adjustbox}
\usepackage{paralist}
\usepackage{wrapfig}
\usepackage{geometry}
 \geometry{
 a4paper,
 total={170mm,257mm},
 left=20mm,
 top=20mm,
 }

%----------------------------------------------------------------------------------------
\newcommand*{\plogo}{\fbox{$\mathcal{PL}$}} % Generic publisher logo

%----------------------------------------------------------------------------------------
%	TITLE PAGE
%----------------------------------------------------------------------------------------

\newcommand*{\titleGP}{\begingroup % Create the command for including the title page in the document
\centering % Center all text
\vspace*{\baselineskip} % White space at the top of the page

\rule{\textwidth}{1.6pt}\vspace*{-\baselineskip}\vspace*{2pt} % Thick horizontal line
\rule{\textwidth}{0.4pt}\\[\baselineskip] % Thin horizontal line

{\LARGE Caracal Automated Marking \\ Software Requirements \\ Specification}\\[0.2\baselineskip]  % Title 

Version: (1.0.0) 


\rule{\textwidth}{0.4pt}\vspace*{-\baselineskip}\vspace{3.2pt} % Thin horizontal line
\rule{\textwidth}{1.6pt}\\[\baselineskip] % Thick horizontal line

\scshape % Small caps
\vspace*{2\baselineskip} % Whitespace between location/year and editors

Edited by \\[\baselineskip]
{\Large Oratile Motswagosele \\ Mankgwanyane Tlaka \\ Lesego Makaleng \\ Kenneth Mangwane \\ Tlou Lebelo\par} % Editor list

{\itshape University of Pretoria\par} % Editor affiliation

\begin{figure}[t]
\centering
	\includegraphics[width=350px]{UP_Logo.PNG}
\end{figure}

{\scshape 2017} \\[0.3\baselineskip] % Year published
{ Project Client: Caracal Research}\par % Publisher

\endgroup}

%----------------------------------------------------------------------------------------
%	BLANK DOCUMENT
%----------------------------------------------------------------------------------------

\begin{document} 

\titleGP % This command includes the title page
\pagebreak

\clearpage
\tableofcontents
\clearpage

\section{Introduction}

E-learning is a topic that is receiving a huge amount of attention worldwide and especially in South Africa. Currently most e-learning in schools consists of taking paper material and making it available on electronic devices such as tablets. \\ 

LMS systems such as Moodle gives the option to design and enter electronic courses and presenting assessments in the form of multiple choice and other one-dimensional methods. The available commerical systems are available but cost is a problem.

\section{Vision}

Caracal Automated Marking is aimed at  developing a metalanguage for the electronic assessment of grade 12 mathematical question in which the steps employed to arrive at the answer can be assessed, not only the final answer. The system should be ideally available as a Moodle plugin. \\

The proposed solution, ideally would be integrated into the Moodle open learning platform as a plugin. The system should accept three documents- i.e., a document with mathematics question; a document with the memorandum and a document with student answers. The system should analyze answer document and give a mark. \\

At the end the proposed  system should ideallly act as methodology through which any given problem and matching memorandum can be entered. 

\section{Background}

E-learning is a topic that receives a lot of attention worldwide and even more in Africa. The problem of educating dispersed people with few resources is a headache that troubles many government and new initiatives are being launched daily to try and solve the problem. \\

Taking a deeper look at the grade 12 question paper marking process, it is realised that most of the time, individuals are still needed who have the knowledge and skill to go through all the answered mathematics scripys and grade them correctly. This process has proven to be time costly and human error tends to creep up at a minimal level.

\section{Architecture requirements}

	\subsection{Access channel requirements}
	%place text here
	\subsection{Quality requirements}	
	%place text here
	\subsection{Integration requirements}
	%place text here
	\subsection{Architecture constraintss}	
	%place text here
\section{Functional requirements and application design}

	\subsection{Use case prioritization}

		\subsubsection{Critical:}
			\begin{itemize}
				\item Register User
				\item Input Question Paper
				\item Input Memo
				\item Input Student Answer Script
				\item Save All Input Files
				\item Compare Student Answer Script against Memo
				\item Persist and save obtained mark
				\item Output Marks
				\item Output Marked Script
			\end{itemize}
		\subsubsection{Important:}
			\begin{itemize}
				\item Validate User Information/Credentials
				\item Validate Question Paper
				\item validate Memo
				\item validate Student Answer Script
				\item Slice Question Paper
				\item Slice Memo
				\item Slice Student Answer Script
				\item Convert Sliced  Question Paper Sections To Suitable Format
				\item Convert Sliced  Memo Sections To Suitable Format
				\item Convert Sliced  Student Answer Script Sections  To Suitable Format
				\item Associate Sliced  Question Paper Section; Memo Section And Student Answer Script Section with Student Details
				
			\end{itemize}
		\subsubsection{Nice-To-Have:}
			\begin{itemize}
				\item Find Student Details From Answer Script
				\item Compile a report which includes information such as highest marks, average, lowest, overall performance etc.
				\item Learn Way Of Answering
				\item Reason For Given Mark
			
			\end{itemize}
	\subsection{Use case/Services contracts}
%For each use case/service specify
		\subsubsection{Pre-Conditions:}
		%place text here
		\subsubsection{Post-Conditions:}
		%place text here
		\subsubsection{Request and Results Data Structures:}
		%place text here
	\subsection{Required functionality}
	%place text here
	\subsection{Process specification}
	%place text here
	\subsection{Domain Model}
	%place text here
	
\section{Open Issues}

\end{document}
	




